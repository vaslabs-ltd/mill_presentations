\documentclass{beamer}

\mode<presentation>
{
  \usetheme{Madrid}
  \usecolortheme{rose}
  \usefonttheme{structurebold}
  \setbeamertemplate{navigation symbols}{}
  \setbeamertemplate{caption}[numbered]
}

\usepackage[english]{babel}
\usepackage[utf8x]{inputenc}
\usepackage{listings}
\usepackage{inconsolata}
\usepackage{qrcode}

\usepackage{tikz}
\usetikzlibrary{arrows.meta, positioning}

\usepackage{upquote}

\lstset{
  basicstyle=\ttfamily\small,
  breaklines=true,
  columns=fullflexible,
  frame=single,
  showstringspaces=false
}

\title[Mill as an Alternative Android Build Tool]{Mill as an Alternative Android Build Tool}
\author{Vasilis Nicolaou}
\institute{Vaslabs LTD}
\date{Android x GDG Manchester winter meetup @ 22 January 2026}

\AtBeginSection[]{
  \begin{frame}
  \vfill
  \centering
  \begin{beamercolorbox}[sep=8pt,center,shadow=true,rounded=true]{title}
    \usebeamerfont{title}\insertsectionhead\par%
  \end{beamercolorbox}
  \vfill
  \end{frame}
}

\begin{document}

% =========================
% Title & Agenda
% =========================

\begin{frame}
  \titlepage
\end{frame}

\begin{frame}{Outline}
  \tableofcontents
\end{frame}

% =========================
% Intro / Context
% =========================

\section{Introduction}

\begin{frame}{Who am I?}
  \begin{itemize}
    \item JVM developer (Java, Scala) with experience in distributed systems.
    \item Contributor to Mill: Android, GraalVM, Spring Boot, Micronaut.
    \item I went through various build tools; Maven, Gradle, Sbt and now... Mill!
    \item Today: Software Engineering Consultant with own firm and a small but strong team of software engineers!
    \item Disclaimer: I'm not an Android developer
  \end{itemize}
\end{frame}

% =========================
% Live Demos First
% =========================

\section{Mill Live Demo}

\subsection{Architecture Samples (Hilt/KSP, Simple Compose)}

\begin{frame}[fragile]{Architecture Samples (BuildConfig, Hilt/KSP, Simple Compose)}
  \vspace{0.5em}
  \textbf{Basic CLI usage:}
  \begin{lstlisting}[language=bash, basicstyle=\ttfamily\tiny]
./mill app.compile
./mill app.androidApk
./mill app.androidTest.androidTestApk
./mill app.createAndroidVirtualDevice
./mill app.startAndroidEmulator
./mill app.androidInstall
./mill app.androidRun --activity com.example.android.architecture.blueprints.todoapp.TodoActivity
./mill app.androidTest
./mill app.stopAndroidEmulator
\end{lstlisting}
\end{frame}

\subsection{Compose Samples}


\begin{frame}[fragile]{JetNews}
  \vspace{0.5em}
  \textbf{Install and test}
  \begin{lstlisting}[language=bash, basicstyle=\ttfamily\tiny]
./mill JetNews.app.androidApk
./mill JetNews.app.androidTest.androidTestApk
./mill JetNews.app.startAndroidEmulator
./mill JetNews.app.androidInstall
./mill JetNews.app.androidTest
\end{lstlisting}

\end{frame}

\begin{frame}[fragile]{Jetcaster}
  \vspace{0.5em}
  \textbf{Start a wear emulator and run}
  \begin{lstlisting}[language=bash, basicstyle=\ttfamily\tiny]
./mill Jetcaster.wear.androidApk
./mill Jetcaster.wear.startAndroidEmulator
./mill Jetcaster.wear.androidInstall
./mill Jetcaster.wear.andoidRun --activity .MainActivity
\end{lstlisting}

\begin{lstlisting}[language=Scala, basicstyle=\ttfamily\tiny]
override def androidVirtualDevice = AndroidVirtualDevice(
  apiVersion = "android-34",
  architecture = "x86_64",
  deviceId = "wearos_large_round",
  systemImageSource = "android-wear"
)
\end{lstlisting}
\end{frame}

\subsection{Mill Programmable syntax}

\begin{frame}[fragile]{Setting up an Android app (plain Java)}
\begin{lstlisting}[language=Scala, basicstyle=\ttfamily\tiny]
package build
import mill.*, androidlib.*, scalalib.*

object androidSdkModule0 extends AndroidSdkModule {
  def buildToolsVersion = "35.0.0"
}

object app extends AndroidAppModule {
  def androidSdkModule = mill.api.ModuleRef(androidSdkModule0)
  def androidMinSdk = 19
  def androidCompileSdk = 35
  def androidApplicationId = "com.helloworld.app"
  def androidApplicationNamespace = "com.helloworld.app"

  object test extends AndroidAppTests, TestModule.Junit4 {
    def junit4Version = "4.13.2"
  }

  object androidTest extends AndroidAppInstrumentedTests {

    def androidSdkModule = mill.api.ModuleRef(androidSdkModule0)

    def mvnDeps = Seq(
      mvn"androidx.test.ext:junit:1.2.1",
      mvn"androidx.test:runner:1.6.2",
      mvn"androidx.test.espresso:espresso-core:3.5.1",
      mvn"junit:junit:4.13.2"
    )
  }
}
\end{lstlisting}
\end{frame}


\begin{frame}[fragile]{Setting up an Android app (Kotlin)}
\begin{lstlisting}[language=Scala, basicstyle=\ttfamily\tiny]
package build
import mill.*, androidlib.*, kotlinlib.*

object androidSdkModule0 extends AndroidSdkModule {
  def buildToolsVersion = "35.0.0"
}

object app extends AndroidAppKotlinModule {
  def kotlinVersion = "2.0.20"
  def androidSdkModule = mill.api.ModuleRef(androidSdkModule0)
  def androidMinSdk = 19
  def androidCompileSdk = 35
  def androidApplicationId = "com.helloworld.app"
  def androidApplicationNamespace = "com.helloworld.app"
  object test extends AndroidAppKotlinTests, TestModule.Junit4 {
    def junit4Version = "4.13.2"
  }
  object androidTest extends AndroidAppKotlinInstrumentedTests {
    def mvnDeps = Seq(
      mvn"androidx.test.ext:junit:1.2.1",
      mvn"androidx.test:runner:1.6.2",
      mvn"androidx.test.espresso:espresso-core:3.5.1",
      mvn"junit:junit:4.13.2"
    )
  }
}
\end{lstlisting}
\end{frame}

\section{Android build process: The tricky parts}

\subsection{Overview}

\begin{frame}[fragile]{Android build plan}
  \begin{figure}
    \centering
    \includegraphics[scale=0.045]{jetnews_visualise_plan}
    \caption{Build tasks and their dependencies}
  \end{figure}
\end{frame}

\subsection{Dependency Resolution}

\begin{frame}[fragile]{Gradle Modules}
  \begin{itemize}
    \item Gradle modules for a multi-platform dependency resolution with variants
  \end{itemize}
  \begin{lstlisting}[language=Java, basicstyle=\ttfamily\tiny]
"attributes": {
  "org.gradle.category": "library",
  "org.gradle.usage": "java-api",
  "org.jetbrains.kotlin.platform.type": "androidJvm"
}
  \end{lstlisting}
\end{frame}

\begin{frame}[fragile]{BOM resolution}
  \begin{itemize}
    \item Use BOMs for compile time and runtime dependencies.
    \item Resolve compiler plugins taking BOMs into account.
  \end{itemize}
  \begin{lstlisting}[language=Scala, basicstyle=\ttfamily\tiny]
def kotlinSymbolProcessorsResolved: T[Seq[PathRef]] = Task {
  defaultResolver().classpath(
    kotlinSymbolProcessors(),
    boms = allBomDeps()
  )
}
  \end{lstlisting}
\end{frame}

\begin{frame}[fragile]{Manage AAR artifacts}
  \begin{itemize}
    \item Extract AARs to get access to resources and AndroidManifest.xml
  \begin{lstlisting}[language=Scala, basicstyle=\ttfamily\tiny]
    case class UnpackedDep(
      name: String,
      classesJar: Option[PathRef],
      repackagedJars: Seq[PathRef],
      proguardRules: Option[PathRef],
      androidResources: Option[PathRef],
      assets: Option[PathRef],
      manifest: Option[PathRef],
      lintJar: Option[PathRef],
      metaInf: Option[PathRef],
      nativeLibs: Option[PathRef],
      baselineProfile: Option[PathRef],
      rTxtFile: Option[PathRef],
      publicResFile: Option[PathRef]
    )
  \end{lstlisting}
    \item Link resources (aapt2 link)
    \item Dynamic/Static resource linking (final/non final IDs)
      \begin{itemize}
        \item Final R IDs i.e. integer constants are inlined into the bytecode (no field access, just an inlined number).
        \item Compile module dependencies with non final IDs (final IDs)
        \item Compile main module with final Ids
      \end{itemize}
    \item Package resources into final APK
  \end{itemize}

\end{frame}

\subsection{Jetpack Compose}

\begin{frame}[fragile]{Jetpack Compose Compiler Plugin}
  \begin{itemize}
    \item Build support for Jetpack compose
    \item Screenshot support with compose preview renderer
    \item IDE integration support is still ongoing
  \end{itemize}
  \begin{block}{Dependency resolution}
    The preview renderer runs locally on a JVM. The libraries required are of a different
    variant than the ones required for Android runtime.
  \end{block}

  \begin{lstlisting}[language=Scala, basicstyle=\ttfamily\tiny]
  private[mill] def addJvmVariantAttributes: ResolutionParams => ResolutionParams = { params =>
    params.addVariantAttributes(
      "org.jetbrains.kotlin.platform.type" -> VariantMatcher.Equals("jvm"),
      "org.gradle.jvm.environment" -> VariantMatcher.Equals("standard-jvm")
    )
  }
  \end{lstlisting}
\end{frame}

\subsection{Dependency Injection with Hilt}

\begin{frame}[fragile]{Hilt / Dagger Annotation Processing}
  \begin{itemize}
    \item Support for Hilt / Dagger annotation processing
    \item Support for Kotlin Symbol Processing (KSP v1 and v2) - which also extended Mill support for other platforms and frameworks
    \item Extra step: Bytecode manipulation with ASM 
  \end{itemize}
\end{frame}

\begin{frame}[fragile]{Bytecode Manipulation with ASM for Hilt Support}
  \begin{lstlisting}[language=Java, basicstyle=\ttfamily\tiny]
.method public constructor <init>()V
    .registers 1

    .line 29
    invoke-direct {p0}, Landroidx/activity/ComponentActivity;-><init>()V

    .line 28
    return-void
.end method
  \end{lstlisting}
  \begin{lstlisting}[language=Java, basicstyle=\ttfamily\tiny]
.method public constructor <init>()V
    .registers 1

    .line 29
    invoke-direct {p0}, Lcom/example/android/architecture/blueprints/todoapp/Hilt_TodoActivity;-><init>()V

    return-void
.end method
  \end{lstlisting}
\begin{block}{References}
  ASM manipulation uses the same tool as Gradle: \textbf{dagger.hilt.android.plugin.transform.AndroidEntryPointClassVisitor}
  Find more details in this PR: \url{https://github.com/com-lihaoyi/mill/pull/4759}
\end{block}
\end{frame}

\subsection{APK size optimisation}

\begin{frame}[fragile]{APK Size Optimisation}
  \begin{itemize}
    \item Support Proguard / R8 rules from dependencies
    \item Support code shrinking with R8
    \item Support resource shrinking (still ongoing)
    \item Most complex: Separate the test APK from the main APK
  \end{itemize}
\end{frame}

\begin{frame}[fragile]{Test APK: exclude dependencies from main APK}
  \begin{lstlisting}[language=Scala, basicstyle=\ttfamily\tiny]
    def androidPackagableDepsExclusionRules: T[Seq[(String, String)]] = Task {
      val baseResolvedDependencies = defaultResolver().resolution(
        Task.traverse(compileModuleDepsChecked)(_.mvnDeps)().flatten,
        boms = allBomDeps()
      )
      baseResolvedDependencies.dependencies
        .map(d => d.module.organization.value -> d.module.name.value).toSeq
    }

    def androidPackagableMvnDeps: T[Seq[Dep]] = Task {
      mvnDeps().map(_.exclude(androidPackagableDepsExclusionRules()*))
    }
    
    override def androidR8CompileOnlyClasspath: T[Seq[PathRef]] = Task {
      Seq(outer.androidR8Jar())
    }
  \end{lstlisting}
  \begin{block}{Summary}
    \begin{itemize}
      \item Use the main APK dependencies as exclusion rules in the test APK dependencies to avoid packaging them.
      \item Compile the main module as a Jar and pass it to the test APK as compile only classpath.
    \end{itemize}
  \end{block}
\end{frame}

\begin{frame}[fragile]{Test APK size comparison}
  \begin{itemize}
    \item Not a competition, we are still chasing the golden example (Gradle).
    \item Shows how close we are in a lot of areas.
  \end{itemize}
\end{frame}
\begin{frame}{Test APK size: Mill}
  \begin{figure}
    \centering
    \includegraphics[width=1\linewidth]{test_apk_mill.png}
    \caption{Mill Test Apk}
  \end{figure}
\end{frame}
\begin{frame}{Test APK size: Gradle}
  \begin{figure}
    \centering
    \includegraphics[width=1\linewidth]{test_apk_gradle.png}
    \caption{Gradle Test Apk}
  \end{figure}
\end{frame}



% =========================
% Comparison & Wrap-up
% =========================

\section{Closing Remarks}

\subsection{Things we are working on}

\begin{frame}{Ongoing Work}
  \begin{itemize}
    \item Improve IDE integration (intelliJ) Support for:
      \begin{itemize}
        \item Compose completions.
        \item Running the app through the IDE.
      \end{itemize}
    \item Support well known third party plugins and frameworks (e.g. Robolectric)
    \item Integrate more examples, more configurations, continue improving.
    \item Get feedback and information from the development community: continuous UX improvements
  \end{itemize}
\end{frame}

\begin{frame}{What's working well}
  \begin{itemize}
    \item Building from the CLI and minimal CI setup.
    \item Mill setups the Android environment, no need for manual user input (apart from accepting those licenses!)
    \item Complex setups: KSP, Hilt, View/Data binding, Jetpack Compose, instrumented tests, Android Native.
  \end{itemize}
\end{frame}

\begin{frame}{How to get started}
  \begin{itemize}
    \item Mill installation: \url{https://mill-build.org/mill/cli/installation-ide.html}
    \item Android setup: \url{https://mill-build.org/mill/android/android-initial-setup.html}
    \item Migration from Gradle flow: \url{https://youtu.be/VL_oPnvkiNY}
    \item Examples (the official compose samples migrated to Mill, with continuous integration): \url{https://github.com/vaslabs-ltd/compose-samples-with-mill}
    \item Blog post: \url{https://mill-build.org/blog/15-android-build-flow.html}
  \end{itemize}
\end{frame}

\subsection{Q\&A}

\begin{frame}{Open Source and Vaslabs}
  \begin{itemize}
    \item Exposure (showing our work, that we can innovate and solve often unique problems)
    \item Small firms often struggle to get talent. We hire undergrads and get them real work in open source.
    \item It's kind of fun!
  \end{itemize}
\end{frame}

\begin{frame}{Questions}
  \begin{itemize}
    \item We are looking to find more about your flow: Daily habits, what kind of features you use daily.
    \item What do you wish Gradle did differently?
    \item Do you use build variants? We don't have a special support specific to Android as Mill kind of supports that out of the box.
  \end{itemize}
\end{frame}

\begin{frame}{Thank you!}
  \centering
  \Large{Thank you!}\\[1em]
  \normalsize
  \begin{itemize}
      \item Slides: \url{https://github.com/vaslabs-ltd/mill_presentations}
      \item Documentation: \url{https://mill-build.org/}
      \item Github Repo: \url{https://github.com/com-lihaoyi/mill}
      \item Try Mill and join Mill's Discord Channel:
  \end{itemize}

  \quad
  \qrcode[height=5cm]{https://discord.gg/bt3DPbfUmV}

\end{frame}

\end{document}
